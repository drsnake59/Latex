\section{Présentation}

	\subsection{Qu'est-ce que le Java ?}
	
		\begin{frame}
			\frametitle{Présentation}
			\framesubtitle{Qu'est-ce que le Java ?}
			\begin{block}{Histoire}
				\begin{itemize}[<+->]
					\item Alan \bsc{Key} conçoit le langage \og SmallTalk\fg{} qui est encore \emph{la} référence dans les langages orientés objets.
					\item Création de nombreuses extensions objet pour le C, C++, Object Pascal, etc.
					\item Au milieu des années 90, Sun publie Java.
				\end{itemize}
			\end{block}
		\end{frame}
	
		\begin{frame}
			\frametitle{Présentation}
			\framesubtitle{Qu'est-ce que le Java ?}
			Le Java est:
			\begin{itemize}[<+->]
				\item un langage orienté objet (POO\footnote{Programmation Orientée Objet})
				\item une architecture\emph{Virtual Machine}
				\item un ensemble d'\emph{API} variées
				\item un ensemble d'outils: le \emph{JDK}\footnote{Java Development Kit}
				\item portable
					\begin{itemize}
						\item \emph{JVM}\footnote{Java Virtual Machine} présente sur systèmes Windows, Mac et Unix
						\item accompagné d'une librairie standard
					\end{itemize}
				\item robuste
					\begin{itemize}
						\item mécanisme d'\emph{exceptions}
					\end{itemize}
			\end{itemize}
		\end{frame}	
	
		\begin{frame}
			\frametitle{Présentation}
			\framesubtitle{Qu'est-ce que le Java ?}
			\begin{alertblock}{Attention !}
				\begin{itemize}
					\item <1-> Le Java n'est pas du JavaScript. Le Java est un langage \og généraliste\fg{}, contrairement au JavaScript qui est orienté sur la programmation Web.
					\item <2-> Le Java n'est pas du C++. Java est un langage \emph{purement} objet et de plus haut niveau.
				\end{itemize}
			\end{alertblock}
		\end{frame}	
	
	\subsection{Les outils}
	
		\begin{frame}
			\frametitle{Présentation}
			\framesubtitle{Les outils}
			Environnements de développement:
			\begin{itemize}
				\item SunJDK
				\item Eclipse
				\item IntelliJ (version \og community\fg{} gratuite, commerciale payante)
				\item NetBeans
			\end{itemize}
			\pause
			\begin{remarque}
				Dans les TPs, nous utiliserons les environnements de développement IntelliJ et Eclipse.
			\end{remarque}
		\end{frame}	
	
		\begin{frame}
			\frametitle{Présentation}
			\framesubtitle{Les outils}
			Liste des outils de Java:
			\begin{description}
				\item[javac:] compilateur de sources Java
				\item[java:] interpréteur de byte code
				\item[appletviewer:] interpréteur d'applet
				\item[javadoc:] générateur de documentation (HTML, MIF)
				\item[javah:] générateur de header pour l'appel de méthodes natives
				\item[javap:] désassembleur de byte code
				\item[jdb:] debugger
				\item[javakey:] générateur de clés pour la signature de code
			\end{description}
		\end{frame}	
	
		\begin{frame}
			\frametitle{Présentation}
			\framesubtitle{Les outils}
			Liste des \emph{API} standards:
			\begin{description}
				\item[java.lang:] types de bases, etc.
				\item[java.util:] HashTable, Vector, Stack, Date...
				\item[java.io:] accès aux entrées/sorties par flux
				\item[java.net:] socket, URL...
				\item[java.sql:] accès homogène aux bases de données
				\item[java.security:] signatures, cryptographie, authentification...
			\end{description}
		\end{frame}	
	
	
	\subsection{Références}
	
		\begin{frame}
			\frametitle{Présentation}
			\framesubtitle{Références}
			Java dispose d'un grand nombre de ressources sur internet.
			
			La version actuelle de Java est la \no 8.
			
			\begin{block}{Documentation officielle}
				\url{https://docs.oracle.com/javase/8/}
			\end{block}
		\end{frame}	